In het vijfde semester van de opleiding Toegepaste Informatica aan de Katholieke Hogeschool Leuven krijgt men het vak ``Beveiliging''. Dit voornamelijk theoretisch vak geeft de studenten een inleiding tot verscheidene veiligheidsproblemen in de Informatica. Ons eindwerk richt er zich op een plaats te voorzien om deze theorie in de praktijk om te zetten en om informatie te voorzien voor studenten die meer willen weten.

In het begin hadden wij de keuze tussen twee mogelijke projecten. Een beveiligingssite of een vliegende drone. Ondanks de interesse in onze ``coole'' vliegende drone hebben we na lang overleggen uiteindelijk toch besloten om voor de beveiligingssite te gaan. Dit mede omdat niemand van ons in het verleden heeft kunnen genieten van een electronica opleiding. Een tweede doorslaggevende factor was dat de drone-modellen op de markt bijna enkel closed-circuit waren, dus die konden we niet gebruiken voor onze doeleinden.

Uiteindelijk was het maken van deze website een leuke en interessante ervaring. Dit omdat we volledige vrijheid hadden in het doen en laten van onze site.