Meneer Geens heeft ons een fysieke server aangeboden die wij volledig geconfigureerd hebben.

De eerste problemen onstonden al na enkele minuten. Deze server kon namelijk niet booten vanaf een USB, dit terwijl er wel degelijk 4 poorten aanwezig zijn. Hierna hebben we Debian Wheezy\cite{wheezy} ge\"installeerd met behulp van een bootable CD. Deze installatie verliep zonder verdere problemen.

Na de installatie hebben wij de sources van OpenVZ\cite{openvz} toegevoegd en OpenVZ ge\"installeerd.
Ook deze installatie verliep van een leie dakje.

Hierna hebben wij een IP-adressen range gekozen voor de verschillende services die moeten draaien op deze server. OpenVZ biedt ons de mogelijkheid verschillende containers\cite{openvzcontainers} te maken die bovenop deze OpenVZ omgeving draaien. Elke service krijgt daarom een eigen container met Debian Wheezy, behalve de container die instaat voor de controle van de ``Memory-based'' oefeningen. Deze service zal draaien onder Ubuntu 7.10\cite{openvzubuntu}.

De keuze voor Ubuntu 7.10 is enkel en alleen omwille van de gcc\cite{gcc} versie (gcc-3.3). Deze gcc-versie houdt nog geen rekening met stack-protection waardoor onze exploit-oefeningen juist gaan functioneren.

\subsection{Host (129)}
NOG UITLEG VAN ADRIAAN OVER DIT SYSTEEM

\subsection{Website (130)}
Deze container draait Debian Wheezy.\\
Apache2\cite{apache} werd hier geïnstalleerd.

Door middel van crontab zal het script ``/root/sync\_website.sh'' elke minuut gaan controleren of er een nieuwe git-commit is geweest. Een positieve git-commit wijst op een lokale aanpassing van de website. Deze lokale aanpassing zal door het script worden overgenomen worden in /var/www. Hierdoor zal de webserver bijna onmiddellijk gesynchroniseerd worden met de lokale website versie.

Een tweede script (/root/sync\_oef.php), dat opgeroepen wordt in het sync\_website.sh script, zal ervoor zorgen dat iedere gebruiker het juiste aantal oefeningen in zijn/haar persoonlijk overzicht heeft staan.
\subsection{SQL (131)}
Deze container draait Debian Wheezy.\\
MySQL werd heir geinstalleerd.

De keuze om de website en de SQL-server in een aparte container te draaien is om de veiligheid van beide containers te garanderen.
\subsection{Coderunner (132)}
NOG UITLEG VAN ADRIAAN OVER DIT SYSTEEM