\subsection{Could not load host key: /etc/ssh/ssh\_host\_ecdsa\_key}
\begin{lstlisting}
Oplossing: sudo ssh-keygen -t ecdsa -f /etc/ssh/ssh_host_ecdsa_key -N 
\end{lstlisting}

\subsection{Opzetten containers}
Apache:
\begin{lstlisting}
root@debian:~# /etc/init.d/apache2 restart
[....] Restarting web server: apache2apache2: apr_sockaddr_info_get() failed for debian
apache2: Could not reliably determine the server's fully qualified domain name, using 127.0.0.1 for ServerName
... waiting apache2: apr_sockaddr_info_get() failed for debian apache2: Could not reliably determine the server's fully qualified domain name, using 127.0.0.1 for ServerName
. ok 
\end{lstlisting}
    
oplossing: /etc/hosts aanpassen:
\begin{lstlisting}
127.0.0.1          debian debian.nwlab.be localhost website localhost.localdomain
192.168.255.130    debian debian.nwlab.be localhost website localhost.localdomain
\end{lstlisting}
   
Error is nu:
\begin{lstlisting}
root@debian:/etc/apache2# /etc/init.d/apache2 restart
[....] Restarting web server: apache2apache2: Could not reliably determine the server's fully qualified domain name, using 127.0.0.1 for ServerName
... waiting apache2: Could not reliably determine the server's fully qualified domain name, using 127.0.0.1 for ServerName
. ok 
\end{lstlisting}
    
/etc/apache2/apache2.conf, regel bijzetten: ServerName localhost

\subsection{Initial config}
\begin{lstlisting}
update-rc.d -f apache2 remove
echo "Naampje" > /etc/hostname
vi /etc/hosts (en zet daarin "Naampje" bij 127.0.0.1)
\end{lstlisting}
    
\subsection{Multiple users die containers aanmaken}
\begin{lstlisting}
Locked by: pid 57388, cmdline vzctl create 136 --ostemplate runcode
Container already locked
Fail. (2304)
Locked by: pid 57388, cmdline vzctl create 136 --ostemplate runcode
Container already locked
ssh: connect to host 192.168.255.136 port 22: No route to host
lost connection
Locked by: pid 57388, cmdline vzctl create 136 --ostemplate runcode
Container already locked
Fail. (2304)
Locked by: pid 57388, cmdline vzctl create 136 --ostemplate runcode
Container already locked
ssh: connect to host 192.168.255.136 port 22: No route to host
lost connection
Locked by: pid 57388, cmdline vzctl create 136 --ostemplate runcode
Container already locked
Locked by: pid 57388, cmdline vzctl create 136 --ostemplate runcode
Container already locked
Locked by: pid 57388, cmdline vzctl create 136 --ostemplate runcode
Container already locked
Locked by: pid 57388, cmdline vzctl create 136 --ostemplate runcode
Container already locked
ssh: connect to host 192.168.255.136 port 22: Connection refused
lost connection
ssh: connect to host 192.168.255.136 port 22: No route to host
lost connection
ssh: connect to host 192.168.255.136 port 22: No route to host
lost connection
Locked by: pid 58387, cmdline vzctl create 136 --ostemplate runcode
Container already locked
Locked by: pid 58387, cmdline vzctl create 136 --ostemplate runcode
Container already locked
Locked by: pid 58387, cmdline vzctl create 136 --ostemplate runcode
Container already locked
Locked by: pid 58387, cmdline vzctl create 136 --ostemplate runcode
Container already locked
Warning: Permanently added '192.168.255.136' (RSA) to the list of known hosts.
Warning: Permanently added '192.168.255.136' (RSA) to the list of known hosts.
^C2014/02/27-11:24:01 Server closing!
\end{lstlisting}
    
1 user die connect (output):\\
r0257589@GT-P323-29:~\$ telnet 192.168.255.132 12345
\begin{lstlisting}
Trying 192.168.255.132...
Connected to 192.168.255.132.
Escape character is '^]'.
userid: 10
oef: 5
code:
#include <stdio.h>
         
int main() {
     printf("Adriaan is zoooooooooooooooooooooo slim");
}
.....
1
Get new container...
Launched container met CTID BEGIN launchve! <-- Die BEGIN is extra output normaal stuurt die enkel de laatste regel door, "136" dus...
Actieve containers: 131 130 134 132 133
Output make_container in launchve: Executing "sudo vzctl create 136 --ostemplate runcode"
Creating container private area (runcode)
Performing postcreate actions
CT configuration saved to /etc/vz/conf/136.conf
Container private area was created
Executing "sudo vzctl set 136 --ipadd 192.168.255.136 --nameserver 192.168.255.254 --save"
CT configuration saved to /etc/vz/conf/136.conf
Made container with CTID 136.
      
Output vzctl start in launchve: Starting container...
Container is mounted
Adding IP address(es): 192.168.255.136
Setting CPU units: 1000
Container start in progress...

136
\end{lstlisting}
    

\subsection{SSH probleem :'(}
\begin{lstlisting}
compile(): ssh -vvvo "StrictHostKeyChecking no" -i /home/adri/.ssh/launchve 
runcode@192.168.255.138 gcc-3.3 /runcode/code.c 2>&1
Output ssh: OpenSSH_6.0p1 Debian-4, OpenSSL 1.0.1e 11 Feb 2013
debug1: Reading configuration data /home/adri/.ssh/config
debug1: /home/adri/.ssh/config line 1: Applying options for 192.168.255.*
debug1: Reading configuration data /etc/ssh/ssh_config
debug1: /etc/ssh/ssh_config line 19: Applying options for *
debug2: ssh_connect: needpriv 0
debug1: Connecting to 192.168.255.138 [192.168.255.138] port 22.
debug1: connect to address 192.168.255.138 port 22: Connection refused
ssh: connect to host 192.168.255.138 port 22: Connection refused
      
-------
      
adri@Exploit:~$ ssh -p 22543 adri@192.168.255.129
Warning: Permanently added '[192.168.255.129]:22543' (ECDSA) to the list of known hosts.
PTY allocation request failed on channel 0
Connection to 192.168.255.129 closed.
      
-------
\end{lstlisting}