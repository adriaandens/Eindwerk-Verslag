Klaas en Quinten hebben zich de voorbije 4 weken vooral geconcentreerd op het maken van de website en de web-based oefeningen. Zo hebben ze geprobeerd deze zo leuk en interactief mogelijk te maken waardoor de studenten aangemoedigd worden om hier veel op te werken. Dit werd gerealiseerd door middel van het implementeren van achievements, eastereggs, highscores, \ldots ~Hierdoor zal een competitie onstaan voor het oplossen van de meeste oefeningen.

Adriaan en Jan hebben het configureren van de server, het maken van de memory-based\cite{taoe} en de CTF oefeningen op hun genomen. Scripts werden geschreven voor allerlei doeleinden. Bijvoorbeeld: het controlleren van de memory-based oefeningen, aanmaken containers, synchronisatie, \ldots

Een specifieke werkverdeling kan worden terug gevonden in ons logboek op onderstaande link:
\url{http://wikis.khleuven.be/sysnw/index.php/Beveiligingssite_logboek}.