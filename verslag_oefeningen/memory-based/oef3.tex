\begin{enumerate}
  \item gcc -g -fno-stack-protector -z execstack -o prism prism.c
  \item gdb -q prism 
  \item (gdb) list
  	\begin{enumerate}
  	\item Door dit commando uit te voeren krijgen we een overzicht van de programma code
  	\item We zoeken de scharnierpunten in dit programma:
  		\begin{enumerate}
  		\item 2 argumenten: \textless IP-address\textgreater \textless password\textgreater
  		\item pw
  		\item return\_val == 0
 		\item Het return adres moet overschreven worden met het adres van de gegevens functie
  		\end{enumerate}
  	\end{enumerate}
  \item set disassembly-flavor intel
  	\begin{enumerate}
  	\item Disassembly taal naar intel zetten
  	\end{enumerate}
  \item disass gegevens
  
  
\end{enumerate}

\begin{lstlisting}
#include <stdio.h>
#include <string.h>
#include <stdlib.h>

int main(int argc, char *argv[]){
	if(argc != 3){
		printf("Error: %s <IP-address> <password> required\n", argv[0]);
		exit(0);
	}

	if(make_connection(argv[1], argv[2]) == 0){
		printf("\nMaking connection...\n");
		printf("====================\n");
		printf("\nConnection failed!\n");
		exit(0);
	}
}

int make_connection(char *ip, char *pass){
	char pw[8];
	int return_val = 0;

	strcpy(pw, pass);

	return return_val;
}

int gegevens(){
	printf("\nNSA\n");
	printf("*********\n");
	printf("Name Password\n");
	printf("------- -----------\n");
	printf("Frans Bauer fRanske123\n");
	printf("Lorrie Popovich loPo007\n");
	printf("Hello Kitty yttikolleh\n");
	printf("Klok Kenluider ringring00\n");

	return(0);
}
\end{lstlisting}