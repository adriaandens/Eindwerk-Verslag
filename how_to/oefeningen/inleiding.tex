\subsection{Web-based} 
\begin{enumerate}
\item `web.php' in de map `application\textbackslash view\textbackslash oefeningen'
	\begin{enumerate}
	\item \$beschrijving = \$array ( \ldots ~, n =\textgreater ``Beschrijving oefening'',);
	\end{enumerate}
\item Bestand `n.php' aanmaken in `application\textbackslash view\textbackslash oefeningen\textbackslash web-oefeningen'
	\begin{enumerate}
	\item Dit bestand wordt de `View' van de oefening
	\end{enumerate}
\item `OefeningenController.php' in de map `application\textbackslash controller'
	\begin{enumerate}
	\item `function oplossing(\$soort, \$nummer)'
	\item `if (\$soort == `web-based')'
	\item Keuze 1
		\begin{enumerate}
		\item \$oplossing = \$array ( \ldots ~, n =\textgreater ``oplossing'',);
		\end{enumerate}
	\item Keuze 2 (Niet controleerbaar met wachtwoord)
		\begin{enumerate}
		\item `if (\$nummer == n) { voeg code hier in }'
		\item Zelf de controle schrijven
		\end{enumerate}
	\end{enumerate}
\item `aantal\_oefeningen' tabel in de database `hacksite'
\item Entry `web', het aantal=`n' veranderen naar `n+1'
\end{enumerate}


\subsection{Memory-based} 
\begin{enumerate}
\item `memory.php' in `application\textbackslash view\textbackslash oefeningen'
\item Toevoegen: \$beschrijving = \$array ( \ldots , n =\textgreater ~``beschrijving oef''
\item In bestand `memory\_oefening.php'
\item \$opgave = \$array ( \ldots , n =\textgreater ~`\textless p class=``uitlijnen''\textgreater Opgavetekst\textless /p\textgreater',);
\item \$bestand = \$array ( \ldots , n =\textgreater ~`link naar bestand',);
\item `OefeningenController.php' in `application\textbackslash controller'
\item `function oplossing(\$soort, \$nummer)'
\item Bij `if (\$soort == ``memory'')' het volgende toevoegen:
\item \$type = \$array (\ldots ,n =\textgreater X',); (x = typenr)
\item Typenr kan het volgende zijn:
	\begin{enumerate}
	\item args
		\begin{enumerate}
		\item Oefening kan opgelost worden door enkel argumenten mee te geven
		\end{enumerate}
	\item code
		\begin{enumerate}
		\item Gebruiker moet zelf code schrijven om oefening op te lossen
		\end{enumerate}
	\end{enumerate}
\item `aantal\_oefeningen' tabel in `hacksite' database:
\item Aanpassen van de entry `memory', aantal=`n' naar `n+1'
\end{enumerate}

\subsection{Forensics} 
\begin{enumerate}
\item Bestand `OefeningenController.php' uit de map `application\textbackslash controller'
\item `function forensics(\$oefening = 0)'
\item `\$tips = \$array (\ldots ,n =\textgreater ~`[`tip1'', ``tip2'']',);
\item In `function oplossing(\$soort, \$nummer)' bij `if (\$soort == ``forensics'')'
\item \$oplossingen = \$array ( \ldots ,n =\textgreater ``oplossing'',);
\item `forensics.php' in `application\textbackslash view\textbackslash oefeningen'
\item \$beschrijving = \$array ( \ldots ,n =\textgreater ~``beschrijving'',); toevoegen
\item  In `forensics\_oefening.php' het volgende toevoegen:
\item \$opgave = \$array ( \ldots , n =\textgreater ~`\textless p class=``vet''\textgreater Opgave\textless /p\textgreater',);
\item Bij `aantal\_oefeningen' tabel in `hacksite' database
\item Entry `forensics', het aantal=`n' naar `n+1' aanpassen
\end{enumerate}

\subsection{Malware} 
\begin{enumerate}
\item `malware.php' in `application\textbackslash view\textbackslash oefeningen'
\item \$beschrijvingen = \$array ( \ldots , n =\textgreater ``beschrijving'',);
\item In bestand `malware\_oefening.php'
\item \$opgave = \$array = ( \ldots , n =\textgreater ` \textless p class=``vet''\textgreater titel opgave.\textless /p\textgreater \textless p\textgreater Uitleg opgave \textless /p\textgreater',);
\item `OefeningenController.php' in `application\textbackslash controller'
\item In `function oplossing(\$soort, \$nummer)' bij `if (\$soort == ``malware'')'
\item \$oplossingen = \$array ( \ldots , n =\textgreater ~``oplossing'',);
\item Bij `aantal\_oefeningen' tabel in `hacksite' database
\item Bij de entry `malware' het aantal = `n' aanpassen naar `n+1'
\end{enumerate}

\subsection{Encryptie} 
\begin{enumerate}
\item `encryptie' tabel in `hacksite'
\item Lijn toevoegen:
	\begin{enumerate}
	\item id = `n'
	\item opgave = `opgave'
	\item tips = `tip1'', ``tip2',
	\item oplossing = `oplossing'
	\item beschrijving = `beschrijving'
	\item extra=`iets extra dat de oefening nodig heeft (vb: afbeelding link)'
	\item `aantal\_oefeningen' tabel in `hacksite' aanpassen, `encryptie' aantal aanpassen naar `n+1'
	\end{enumerate}
\end{enumerate}

\subsection{Trivia categorie} 
\begin{enumerate}
\item `trivia' tabel in 'hacksite'
\item Lijn invoegen
	\begin{enumerate}
	\item id = `n'
	\item beschrijving = `Vragenreeks n'
	\end{enumerate}
\item `aantal\_oefeningen' tabel in `hacksite' aanpassen n+1 bij de entry `trivia'
\end{enumerate}

\subsection{Trivia vragen} 
\begin{enumerate}
\item `trivia\_vragen' tabel in `hacking' database
\item lijn toevoegen
	\begin{enumerate}
	\item id=`n' (= oefnummer in reeks)
	\item trivia\_id=`n' (= gewenste reeks)
	\item keuzes=`keuze1'', ``keuze n'
	\item oplossing=`juiste keuze'
	\end{enumerate}
\end{enumerate}

\subsection{Capture the flag}
\begin{enumerate}
\item `ctf.php' in `application\textbackslash view\textbackslash oefeningen'
\item \$beschrijvingen = \$array ( \ldots , n =\textgreater ~``beschrijving'',
\item Lijn toevoegen in: `ctf\_oefeningen.php'
\item \$beschrijvingen = \$array ( \ldots , ‘ \textless p class=``uitlijnen''\textgreater Uitleg opgave.\textless /p\textgreater \textless p Uitleg opgave \textless /p\textgreater \textless p class=``vet uitlijnen''\textgreater Opgave\textless /p\textgreater \textless p\textgreater  bestand link\textless /p\textgreater ',',
\item `OefeningenController.php' in `application\textbackslash controller'
\item `function oplossing(\$soort, \$nummer)' toevoegen:
\item `if (\$soort = ``ctf'') in de oplossingen array n =\textgreater ``oplossing'','
\item `aantal\_oefeningen' tabel in `hacksite' aanpassen n+1 bij de entry `ctf'
\end{enumerate}

\subsection{Complex}
\begin{enumerate}
\item `complex.php' in `application\textbackslash view\textbackslash oefeningen'
\item \$beschrijvingen = \$array( \ldots , n =\textgreater ~`\textless p class=``uitlijnen''\textgreater Uitleg opgave.\textless /p\textgreater ~\textless p\textgreater Uitleg opgave \textless /p\textgreater,
\item \$urls = \$array ( \ldots , n =\textgreater ~`/complex/'. \$this-\textgreater oefening. `/',
\item De tips: \$oplossingen = \$array ( \ldots , 1 =\textgreater ~``\textless li\textgreater tip\textless /li\textgreater'',);
\item `OefeningenController.php' in `application\textbackslash controller'
\item `function oplossing(\$soort, \$nummer)
\item Toevoegen: `if (\$soort == ``complex'') in de oplossingen array n =\textgreater ``oplossing'',
\item Map `n' aanmaken in `complex' (is de omgeving van de oefening)
\item `aantal\_oefeningen' tabel in `hacksite' bij de entry `ctf' het aantal verhogen (n + 1)
\end{enumerate}

\subsection{Tutorial} 
\begin{enumerate}
\item `n.php' in `application\textbackslash view\textbackslash toturials'
\item `index.php' in dezelfde map: `n' toevoegen
	\begin{enumerate}
	\item \$titel = \$array( \ldots ~n =\textgreater ``titel / beschrijving'',);
	\end{enumerate}
\end{enumerate}

\subsection{Downloadable}
\begin{enumerate}
\item `downloads' tabel in `hacksite' database
\item Entry toevoegen
\item id=`n', titel=`titel', description=`beschrijving downloadable', link=`link naar locatie'
\end{enumerate}

\subsection{Link}
\begin{enumerate}
\item `links.php' in `application\textbackslash view\textbackslash \_partials'
\item Lijn toevoegen in: \textless ul\textgreater
\item \textless li\textgreater \textless img src=``/images/link\_icon.png'' width=``10'' height=``10''\textgreater \textless a href=``link naar pagina''\textgreater naam van site\textless /li\textgreater \textless/a\textgreater 
\end{enumerate}