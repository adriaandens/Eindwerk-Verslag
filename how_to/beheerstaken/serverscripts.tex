De volgende scripts staan in voor de synchronisatie van de lokale website met de publieke website door middel van git.
\subsection{Webserver (192.168.255.130)}
\begin{lstlisting}
#!/bin/bash
#      
#	/root/sync_website.sh
#	----------------------
#
#	Dit eenvoudig script zal ervoor zorgen dat de git-commits,
#	met 1 minuut vertraging, worden doorgevoerd naar de website.
#
#	Met behulp van crontab loopt dit script elke minuut.
#
#	Eerst en vooral wordt er gekeken naar de /home/quinten/Eindwerk_Website
#	folder. We voeren hier het commando: "git pull" uit. Wanneer de map 
#	reeds up-to-date is stopt het script met uitvoeren.
#	Het up-to-date zijn van deze map wil zeggen dat er geen nieuwe commits
#	zijn geweest sinds de laatste commit, dus de /var/www-map is nog steeds
#	de nieuwste versie.
#
#	Wanneer "git pull" niet up-to-date is, en er dus aanpassingen zijn
#	doorgevoerd, moeten wij de website up-to-date brengen.
#
#	Dit up-to-date brengen zal gebeuren door middel van rsync.
#	Het commando "rsync -r --delete ..." zal ervoor zorgen dat de
#	/var/www map gesyncroniseerd wordt met de 
#	/home/quinten/Eindwerk-Website map.
#
#	Omdat lokaal aan deze website gewerkt wordt bevat het config.php bestand
#	de lokale gegevens in plaats van de server gegevens. Daarom zal de
#	juiste config-file bijgehouden worden in een /root/www-backup map.
#	Na iedere succesvolle "git pull" zal deze config.php file dan ook de 
#	lokale config.php file overschrijven in /var/www/application/.
#
#	Door het gebruik van rsync zal de .git map ook gesynchroniseerd worden.
#	Dit mag niet. Deze .git map zal in dit script verwijderd worden van de 
#	/var/www map.
#
#	Database dumps worden ook bijgehouden in git. Deze dienen voor eventuele
#	recovery van de databases. Rsync zal ook deze dumps synchroniseren met de
#	/var/www map. We willen niet dat de gebruikers aan deze dumps kunnen, dus
#	worden deze verwijderd uit de /var/www map.
#
#	Onze hacking.ova zal bij iedere rsync verwijderd worden uit de /var/www
#	map. Hierdoor zullen wij dit bestand ook bijhouden in de /root/www-backup map
#	en opnieuw importeren wanneer nodig.
#
#	Na het importeren worden alle rechten juist gezet door middel van het 
#	chown -R quinten:www-data /var/www/ commando.
#
#	Tenslotte voeren we het /root/sync_oef.php script uit om ervoor te zorgen
#	dat iedere gebruiker bij de overzicht-pagina het juiste aantal oefeningen
#	zal zien.
   
cd /home/quinten/Eindwerk-Website
    
if  `git pull` =~ "up-to-date" 
then
# git is up-to-date
	exit 0;
fi
    
# git is not up-to-date
# sync /home/quinten/Eindwerk-Website with /var/www/
rsync -r --delete /home/quinten/Eindwerk-Website/ /var/www/
    
# Restore current config
cp /root/www_backup/config.php /var/www/application/
    
# Remove the .git directory
rm -r /var/www/.git
    
# Remove Database dumps
rm -r /var/www/database\ dumps/
    
# Add hacking.ova to /var/www/bestanden/
cp /root/www_backup/hacking.ova /var/www/bestanden/
    
# Rights back to quinten
chown -R quinten:www-data /var/www/
    
# sync user overzicht
php -f /root/sync_oef.php
\end{lstlisting}

\subsection{SQL-server (192.168.255.131)}
\begin{lstlisting}
#
#	/root/sync_oef.php
#	------------------
#
#	Dit script zal gebruikt worden om het aantal oefeningen in het  persoonlijk 
#	overzicht per gebruiker te updaten.
#
#	Eerst wordt er een verbinding gemaakt met de MySQL-server op IP-adres
#	192.168.255.131 met de "hacksite" databse met de volgende 
#	credentials: quinten & netniuqvda.
#
#	Of een gebruiker de oefening al dan niet opgelost heeft wordt als volgt
#	bijgehouden in de database per gebruiker per categorie (vb: web-based): 
#		"jjjjnnnnn------------".
#			j = oefening is reeds opgelost
#			n = oefening is nog niet opgelost
#			j + n = totaal aantal oefeningen die op de site staan
#			- = deze oefening bestaat nog niet
#
#	De tabel aantal_oefeningen zal voor iedere categorie het aantal
#	oefeningen bijhouden.
#
#	Het script zal de string "jjjnnnn----..." verder aanvullen met "n" tot
#	de som van het aantal "j" en "n" gelijk is aan de waarde die terug-
#	gevonden wordt in de aantal_oefeningen tabel.
    
<?php
    
   	mysql_connect("192.168.255.131", "quinten", "netniuqvda", "hacksite");
    
   	$query = "
   		SELECT *
   		FROM hacksite.aantal_oefeningen
   	";
          
    $result = mysql_query($query);
   	$oefeningen = array();
    
   	while ($oefening = mysql_fetch_assoc($result)) {
   		$oefeningen[$oefening['soort']] = $oefening['aantal'];
   	}
    
    $query = "
    	SELECT *
    	FROM hacksite.resultaten
    ";
         
    $result = mysql_query($query);
         
    while ($resultaten = mysql_fetch_assoc($result)) {
    	$gebruikersnaam = $resultaten['gebruikersnaam'];
    
    	foreach ($oefeningen as $soort => $aantal_oefeningen) {
    		$reeks = $resultaten[$soort];
    
    		for ($i = 0; $i < $aantal_oefeningen; $i++) {
    			if ($reeks[$i] == '-') {
    				$reeks[$i] = 'n';
    			}
    		}
    
    		$query = "
    			UPDATE hacksite.resultaten
    			SET " . $soort . " = '" . $reeks . "'
    			WHERE gebruikersnaam = '" . $gebruikersnaam . "';     
    	    ";
         
    		mysql_query($query);
    	}
	}
?>
\end{lstlisting}